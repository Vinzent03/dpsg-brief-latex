\documentclass[11pt, parskip=full]{scrlttr2}
\usepackage{carlito}
\usepackage{array}
\usepackage{tabularx}
\usepackage{graphicx}
\usepackage{tikz}
\usepackage[placement=top, angle=5]{background}
\usepackage{xcolor}
\usepackage[automark,headsepline=.3pt,footsepline=.3pt]{scrlayer-scrpage}
\usepackage[a4paper,left=25mm,right=25mm,top=25mm,bottom=25mm,includeheadfoot]{geometry}
\usepackage[ngerman]{babel}
\usepackage{lmodern}
\usepackage{tabularx}
\usepackage{blindtext}
\renewcommand*\familydefault{\sfdefault}

\backgroundsetup{contents={\includegraphics[width=6.05cm, height=6.12cm]{lilie.pdf}}, scale=1, opacity=0.5, position={-0.7cm,-1.8cm}}

\definecolor{dpsgbeige}{RGB}{199,189,173}
\definecolor{dpsgblue}{RGB}{0,48,86}



\setkomavar{location}{
    \begin{minipage}{0.3\textwidth}

        \fontsize{8}{10}\selectfont

        \textbf{Stammesvorstand}\\

        
        \begin{tabularx}{\linewidth}{l>{\raggedright\arraybackslash}X }
        Name &  Max Musterman\\
    \end{tabularx}
\end{minipage}
}

\makeatletter
\@setplength{locvpos}{12cm} % Adjust the vertical position from the top
\@setplength{lochpos}{2,1cm} % Adjust the horizontal position form the right
\@setplength{refvpos}{8.5cm} % Adjust the vertical position of the subject
\makeatother


\addtokomafont{pageheadfoot}{\small\upshape\color{white}}% <- footer linie weiß damit man sie nicht sieht

\setkomavar{subject}{Betreff}
\begin{document}

\begin{letter}{Adresse}
	% ----- Media -----
	\tikz [remember picture, overlay] %
	\node [shift={(-1cm,-1cm)}] at (current page.north east) %
	[anchor=north east] %
	{\includegraphics[width=6cm]{logo.png}};

	\tikz [remember picture, overlay]
	\node [shift={(-198mm,-124mm)}] at (current page.north east) %
	[anchor=north east] %
	{\includegraphics{start.png}};

	\tikz [remember picture, overlay]
	\node [shift={(-11mm, 7mm)}] at (current page.south east) %
	[anchor=south east] %
	{\includegraphics{end.png}};
	% ----- Media -----


	\opening{Anrede,}

	\begin{minipage}{0.73\textwidth}
		

% Füge hier Text für die Titelseite ein

	\end{minipage}

	\pagebreak

	

% Füge hier den Inhalt ab der zweiten Seite ein.


	\closing{Gut Pfad,}

\end{letter}
\end{document}
